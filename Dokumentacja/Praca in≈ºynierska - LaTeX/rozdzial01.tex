\chapter{Wstęp}
	\section{Wprowadzenie}
	Robotyka jest obecnie prężnie rozwijającą się dziedziną nauki. Niskie ceny mikrokontrolerów oraz duża konkurencyjność firm na rynku powodują przenikanie urządzeń robotycznych z zastosowań specjalnych do życia codziennego. Urządzenia te, mogą oszczędzać zasoby ludzkie w codziennych prostych czynnościach.  Dodatkowo projekty takie jak \texttt{Arduino} \cite{arduinoFramework} pozwalają na tworzenie tych urządzeń bez specjalistycznej wiedzy. Kolejnym czynnikiem dynamizującym popularyzację automatyzacji i robotyki jest stopniowe wprowadzanie sieci 5G \cite{5ggov}. Pozwoli ona na wykorzystanie potencjału IoT oraz znaczną automatyzację działania urządzeń robotycznych. W tej pracy skupiono się na budowie urządzenia wspomagającego prace sprzątające.
	
	Na rynku istnieje wiele konstrukcji robotów sprzątających. Skupiają się one głównie na pracy jako odkurzacze. Natomiast liczba autonomicznych robotów myjących podłogi jest zdecydowanie mniejsza. Głównie są to proste roboty mopujące. Robot przedstawiony w pracy ma za zadanie wypełnić lukę między małymi i prostymi urządzeniami, a dużymi do zastosowań profesjonalnych.
	
	\section{Cel i zakres pracy}
	Celem jest zaprojektowanie, zbudowanie i zaprogramowanie autonomicznego robota myjącego podłogi do zastosowań niekomercyjnych.
	
	Rozdział pierwszy opisuje wstęp teoretyczny

\chapter{Wstęp teoretyczny}
\texttt{Robot mobilny} - robot, który potrafi zmieniać swoje położenie w przestrzeni. Może być robotem autonomicznym, czyli takim który realizując swoje zadanie porusza się bezkolizyjnie w wyznaczonym środowisku oraz robi to bez ingerencji operatora.
Roboty mobilne można podzielić na kategorie przedstawione w tabeli~\ref{tab:kategorieRobotow}.
\begin{table}[htb] \small
	\centering
	\caption{Kategorie robotów mobilnych}
	\label{tab:kategorieRobotow}
	\begin{tabularx}{\linewidth}{|c|c|X|p{6cm}|} \hline\
		Narzędzie & Wersja & Opis & Adres \\ \hline\hline
		MiKTeX & 2.9 & Zalecana jest instalacja \texttt{Basic MiKTeX} z dystrubucji 32 lub 64 bitowej. Brakujące pakiety będą się doinstalowywać podczas kompilacji projektu. &
		\url{http://miktex.org/download} \\ \hline
		TexnicCenter & 2.02 &  Można pobrać 32 lub 64 bitową wersję & \url{http://www.texniccenter.org/download/} \\ \hline
		SumatraPDF & 3.1.1 & Można pobrać 32 lub 64 bitową wersję & \url{http://www.sumatrapdfreader.org/download-free-pdf-viewer.html} \\ \hline
		JabRef & 3.3 & Można pobrać 32 lub 64 bitową wersję & \url{http://www.fosshub.com/JabRef.html} \\ \hline
	\end{tabularx}
\end{table}
%%%
%%%Uwaga: tytuł powinien zmieścić się w okienku kolorowej okładki (którą
%%%powinna dostarczyć uczelniana administracja). Proszę posterować
%%%parametrami, aby "wpasować" w okienko własny tekst.
%%%
%%%Do ASAPa należy wprowadzić pracę dyplomową/projekt inżynierski w pliku o nazwie:
%%%
%%%W04_[nr albumu]_[rok kalendarzowy]_[rodzaj pracy] (szczegółowa instrukcja pod adresem asap.pwr.edu.pl)
%%%
           %%%Przykładowo:
        %%%­W04_123456_2015_praca inżynierska.pdf     - praca dyplomowa inżynierska
        %%%W04_123456_2015_projekt inżynierski.pdf   - projekt inżynierski
        %%%W04_123456_2015_praca magisterska.pdf  - praca dyplomowa magisterska
%%%
              %%%rok kalendarzowy ? rok realizacji kursu „Praca dyplomowa” (nie rok obrony) 