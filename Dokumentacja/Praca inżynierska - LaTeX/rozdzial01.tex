

\chapter{Wstęp teoretyczny}
	\texttt{Robot mobilny} - robot, który potrafi zmieniać swoje położenie w przestrzeni. Może być robotem autonomicznym, czyli takim który realizując swoje zadanie porusza się bezkolizyjnie w wyznaczonym środowisku oraz robi to bez ingerencji operatora.
	
	Roboty mobilne można podzielić ze względu na ich mobliność:
	\begin{itemize}
		\item kołowe
		\item kroczące
		\item latające
		\item pływające
	\end{itemize}
	Z kolei w robotach kołowych możemy rozróżnić następujące klasy:
	\begin{itemize}
		\item (3,0) - robot posiadający 3 koła szwedzkie. Najczęściej spotykany w formie trójkątnej platformy z przymocowanymi kołami do wierzchołków trójkąta.
		\item (2,1)
		\item (2,0)
		\item (1,2)
		\item (1,1)
	\end{itemize}

%%%
%%%Uwaga: tytuł powinien zmieścić się w okienku kolorowej okładki (którą
%%%powinna dostarczyć uczelniana administracja). Proszę posterować
%%%parametrami, aby "wpasować" w okienko własny tekst.
%%%
%%%Do ASAPa należy wprowadzić pracę dyplomową/projekt inżynierski w pliku o nazwie:
%%%
%%%W04_[nr albumu]_[rok kalendarzowy]_[rodzaj pracy] (szczegółowa instrukcja pod adresem asap.pwr.edu.pl)
%%%
           %%%Przykładowo:
        %%%­W04_123456_2015_praca inżynierska.pdf     - praca dyplomowa inżynierska
        %%%W04_123456_2015_projekt inżynierski.pdf   - projekt inżynierski
        %%%W04_123456_2015_praca magisterska.pdf  - praca dyplomowa magisterska
%%%
              %%%rok kalendarzowy ? rok realizacji kursu „Praca dyplomowa” (nie rok obrony) 