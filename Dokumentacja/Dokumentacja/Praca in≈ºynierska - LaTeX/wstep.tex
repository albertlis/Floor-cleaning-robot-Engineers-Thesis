\chapter{Wstęp}
	\section{Wprowadzenie}
	Robotyka jest obecnie prężnie rozwijającą się dziedziną nauki. Niskie ceny mikrokontrolerów oraz duża konkurencyjność firm na rynku powodują przenikanie urządzeń robotycznych z zastosowań specjalnych do życia codziennego. Urządzenia te, mogą oszczędzać zasoby ludzkie w codziennych prostych czynnościach.  Dodatkowo projekty takie jak \texttt{Arduino} \cite{arduinoFramework} pozwalają na tworzenie tych urządzeń bez specjalistycznej wiedzy. Kolejnym czynnikiem dynamizującym popularyzację automatyzacji i robotyki jest stopniowe wprowadzanie sieci 5G \cite{5ggov}. Pozwoli ona na wykorzystanie potencjału IoT oraz znaczną automatyzację działania urządzeń robotycznych. W tej pracy skupiono się na budowie urządzenia wspomagającego prace sprzątające.
	
	Na rynku istnieje wiele konstrukcji robotów sprzątających. Skupiają się one głównie na pracy jako odkurzacze. Natomiast liczba autonomicznych robotów myjących podłogi jest zdecydowanie mniejsza. Głównie są to proste roboty mopujące. Robot przedstawiony w pracy ma za zadanie wypełnić lukę między małymi i prostymi urządzeniami, a dużymi do zastosowań profesjonalnych.
	
	\section{Cel i zakres pracy}
	Celem jest zaprojektowanie, zbudowanie i zaprogramowanie autonomicznego robota myjącego podłogi do zastosowań niekomercyjnych.
	Robot powinien:
	\begin{enumerate}
		\item Posiadać wirujące szczotki myjące podłogi
		\item Posiadać system dozujący wodę z detergentem
		\item Posiadać system zbierający zużytą wodę z podłogi
		\item Potrafić rozpoznawać i omijać przeszkody
		\item Potrafić rozpoznawać niebezpieczne różnice wysokości podłogi
		\item Posiadać system jezdny pozwalający łatwo osiągać zadane położenia
		\item Posiadać możliwość określania swojej bieżącej pozycji
		\item Sygnalizować zdarzenia nadzwyczajne
		\item Posiadać system ładowania baterii
	\end{enumerate}
	