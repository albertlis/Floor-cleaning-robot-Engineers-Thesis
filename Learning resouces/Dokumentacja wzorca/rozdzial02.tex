\chapter{Praca z szablonem}
\section{Organizacja szablonu}
Szablon składa się z pliku głównego, plików z kodem kolejnych rozdziałów i dodatków (włączanych do kompilacji w dokumencie głównym), katalogów z plikami grafik (włączanymi do rysunków w rozdziałach), pliku ze skrótami (opcjonalny), pliku z danymi bibliograficznymi (plik \texttt{dokumentacja.bib}). Taki ,,układ'' zapewnia porządek oraz pozwala na selektywną kompilację rozdziałów. Wyjaśniając to dokładniej, podczas tworzenia szablonu przyjęto następującą konwencję:
\begin{itemize}
\item Plikiem głównym jest plik \texttt{Dyplom.tex}. To w nim znajdują się deklaracje wszystkich używanych styli, definicje makr oraz ustawień, jak również polecenie \verb+\begin{document}+.
\item Teksty rozdziałów są redagowane w osobnych plikach o nazwach zawierających numer rozdziału. Pliki te zamieszczone są w katalogu głównym (tym samym, co plik \texttt{Dyplom.tex}). I~tak \texttt{rozdzial01.tex} to plik pierwszego rozdziału (ze Wstępem), \texttt{rozdzial02.tex} to plik z treścią drugiego rozdziału itd. 
\item Teksty dodatków są redagowany w osobnych plikach o nazwach zawierających literę dodatku. Pliki te, podobnie do plików z tekstem rozdziałów, zamieszczane są w katalogu głównym. I~tak \texttt{dodatekA.tex} oraz \texttt{dodatekB.tex} to, odpowiednio, pliki z treścią dodatku A oraz dodatku B.
\item Pewnym wyjątkiem od reguły nazewniczej w przypadku plików z tekstem rozdziałów i dodatków jest plik \texttt{skroty.tex}. Jest to plik, w którym zamieszczono wykaz użytych skrótów. W jego nazwie nie występuje żaden numer czy porządkowa litera. 
\item Każdemu rozdziałowi i dodatkowi towarzyszy katalog przeznaczony do składowania dołączanych w nim grafik. I tak \texttt{rys01} to katalog na pliki z grafikami dołączanymi do rozdziału pierwszego, \texttt{rys02} to katalog na pliki z grafikami dołączanymi do rozdziału drugiego itd.
Podobnie \texttt{rysA} to katalog na pliki z grafikami dołączanymi w dodatku A itd.
\item W katalogu głównym zamieszczany jest plik \texttt{dokumentacja.bib} zawierający bazę danych bibliograficznych.
\end{itemize}

\begin{table}[htb]
\centering\small
\caption{Pliki źródłowe szablonu oraz wyniki kompilacji}
\label{tab:szablon}
\begin{tabularx}{\linewidth}{|p{.55\linewidth}|X|}\hline
Źródła & Wyniki kompilacji \\ \hline\hline
\verb?Dokument.tex? - dokument główny\newline
\verb?Dokument.tcp? -- szablon projektu \texttt{MiKTeX}\newline
\verb?rozdzial01.tex? -- plik rozdziału \texttt{01}\newline
\verb?...?\newline
\verb?dodatekA.tex? -- plik dodatku \texttt{A}\newline
\verb?...?\newline
\verb?rys01? -- katalog na rysunki do rozdziału \texttt{01}\newline
\verb?   |- fig01.png? -- plik grafiki\newline
\verb?   |- ...?\newline
\verb?...?\newline
\verb?rysA? -- katalog na rysunki do dodatku \texttt{A}\newline
\verb?   |- fig01.png? -- plik grafiki\newline
\verb?   |- ...?\newline
\verb?...?\newline
\verb?dokumentacja.bib? -- plik danych bibliograficznych\newline
\verb?Dyplom.ist? -- plik ze stylem indeksu\newline
\verb?by-nc-sa.png? -- plik z ikonami CC\newline
 &
\verb?Dyplom.bbl?\newline
\verb?Dyplom.blg?\newline
\verb?Dyplom.ind?\newline
\verb?Dyplom.idx?\newline
\verb?Dyplom.lof?\newline
\verb?Dyplom.log?\newline
\verb?Dyplom.lot?\newline
\verb?Dyplom.out?\newline
\verb?Dyplom.pdf? -- dokument wynikowy\newline
\verb?Dyplom.syntex?\newline
\verb?Dyplom.toc?\newline
\verb?Dyplom.tps?\newline
\verb?*.aux?\newline 
\verb?Dyplom.synctex?\newline\\
\hline
\end{tabularx}
\end{table}

Szablon przygotowano w systemie Windows stosując kodowanie \texttt{cp1250}. Można go wykorzystać również w innych systemach i przy innych kodowaniach. Jednakże wtedy konieczna jest korekta dokumentu \texttt{Dyplom.tex} odpowiednio do wybranego przypadku. Korekta ta polegać może na zamianie polecenia \verb+\usepackage[cp1250]{inputenc}+  na polecenie \verb+\usepackage[utf8]{inputenc}+ oraz konwersji kodowania istniejących plików ze źródłem latexowego kodu (plików o rozszerzeniu \texttt{*.tex} oraz \texttt{*.bib}).

Samo kodowanie plików może być źródłem paru problemów. Chodzi o to, że użytkownicy pracujący z edytorami tekstów pod linuxem mogą generować pliki zakodowane w UTF-8 bez BOM lub z BOM, a pod windowsem -- pliki z kodowaniem znaków \texttt{cp1250} zakodowanych w~ANSI. A z takimi plikami różne edytory różnie sobie radzą (w szczególności edytor TeXnicCenter czasami z niewiadomego powodu traktuje zawartość pliku jako UTF8 lub ANSI -- chyba sprawdza, czy w bufore nie ma jakichś znaków specjalnych i na tej podstawie interpretuje kodowanie). Bywa, że choć wszystko wygląda OK to jednak kompilacja latexowa ,,nie idzie''. Problemem mogą być właśnie pierwsze bajty, których nie widać w edytorze. 

Kodowanie znaków jest istotne również przy edytowaniu bazy danych bibliograficznych (pliku \texttt{dokumentacja.bib}). Aby \texttt{bibtex} poprawnie interpretował polskie znaki plik \texttt{dokumentacja.bib} powinien być zakodowany w ANSI, CR+LF (dla ustawień jak w szablonie). Do konwersji kodowania można użyć Notepad++ (jest tam opcja ,,konwertuj'' - nie mylić z opcją ,,koduj'', która przekodowuje znaki, jednak nie zmienia sposobu kodowania pliku).



\section{Kompilacja szablonu}
Kompilację szablonu może uruchamić na killka różnych sposobów. Wszystko zależy od używanego systemu operacyjnego, zaintalowanej na nim dystrybucji latexa oraz dostępnych narzędzi. Zazwyczaj kompilację rozpoczyna się wydając polecenie z linii komend lub uruchamia się ją za pomocą narzędzi zintegrowanych środowisk.

Kompilacja z linii komend polega na uruchomieniu w katalogu, w którym rozpakowano źródła szablonu, następującego polecenia:
\begin{lstlisting}[basicstyle=\ttfamily]
> pdflatex Dyplom.tex
\end{lstlisting}
gdzie \texttt{pdflatex} to nazwa kompilatora, zaś \texttt{Dyplom.tex} to nazwa głównego pliku redagowanej pracy. 
W przypadku korzystania ze środowiska \texttt{TeXnicCenter} należy otworzyć dostarczony w szablonie plik projektu \texttt{Dyplom.tcp}, a następnie uruchomić kompilację narzędziami dostępnymi w pasku narzędziowym.

Aby poprawnie wygenerowały się wszystkie referencje (spis treści, odwołania do tabel, rysunków, pozycji literaturowych, równań itd.) kompilację \texttt{pdflatex} należy wykonać dwukrotnie, a~czasem nawet trzykrotnie, gdy wygenerowane mają zostać odwołania do pozycji literaturowych oraz wykazu literatury. 

Wygenerowanie danych bibliograficznych zapewnia kompilacja \texttt{bibtex} uruchamiana po kompilacji \texttt{pdfltex}. Można to zrobić z linii komend:
\begin{lstlisting}[basicstyle=\ttfamily]
> bibtex Dyplom
\end{lstlisting}
lub wybierając odpowiednią pozycję z paska narzędziowego wykorzystywanego środowiska. Po kompilacji \texttt{bibtex} na dysku pojawi się plik \texttt{Dyplom.bbl}. Dopiero po kolejnych dwóch kompilacjach \texttt{pdflatex} dane z tego pliku pojawią się w wygenerowanym dokumencie. Podsumowując, po każdym wstawieniu nowego cytowania w kodzie dokumentu uzyskanie poprawnego formatowania dokumentu wynikowego wymaga powtórzenia następującej sekwencji kroków kompilacji:
\begin{lstlisting}[basicstyle=\ttfamily]
> pdflatex Document.tex
> bibtex Document
> latex Document.tex
> latex Document.tex
\end{lstlisting}
Szczegóły dotyczące przygotowania danych bibliograficznych oraz zastosowania cytowań przedstawiono w podrozdziale \ref{sec:literatura}.

W głównym pliku zamieszczono polecenia pozwalające sterować procesem kompilacji poprzez włączanie bądź wyłączanie kodu źródłowego poszczególnych rozdziałów. Włączanie kodu do kompilacji zapewniają instrukcje \verb+\include+ oraz \verb+\includeonly+. Pierwsza z nich pozwala włączyć do kompilacji kod wskazanego pliku (np.\ kodu źródłowego pierwszego rozdziału \verb+\chapter{Wstęp}
\section{Wprowadzenie}
Niniejszy dokument powstał z myślą o ujednoliceniu sposobu redagowania prac dyplomowych. Jego źródła mają pełnić rolę szablonu nowoedytowanej pracy, zaś treść powinna być interpretowana jako zestaw zaleceń i uwag o charakterze technicznym (dotyczących takich zagadnień, jak: formatowanie tekstu, załączanie rysunków, układ strony itp.) oraz stylistycznym (odnoszących się do stylu wypowiedzi, sposobów tworzenia referencji itp.).

Szablon przygotowano do kompilacji pdflatexem w konfiguracji: \texttt{MiKTeX} (windowsowa dystrybucja latexa) + \texttt{TeXnicCenter} (środowisko do edycji i kompilacji projektów latexowych) + \texttt{SumatraPDF} (przeglądarka pdfów z nawigacją zwrotną) + JabRef (opcjonalny edytor bazy danych bibliograficznych). Jest to zalecany zestaw narzędzi do edycji pracy w systemie Windows. Można je pobrać ze stron internetowych, których adresy zamieszczono w tabeli~\ref{tab:narzedzia}.
\begin{table}[htb] \small
\centering
\caption{Wykaz zalecanych narzędzi do kompilacji szablonu (adresy internetowe ważne na dzień 1.04.2016)}
\label{tab:narzedzia}
\begin{tabularx}{\linewidth}{|c|c|X|p{6cm}|} \hline\
Narzędzie & Wersja & Opis & Adres \\ \hline\hline
MiKTeX & 2.9 & Zalecana jest instalacja \texttt{Basic MiKTeX} z dystrubucji 32 lub 64 bitowej. Brakujące pakiety będą się doinstalowywać podczas kompilacji projektu. &
\url{http://miktex.org/download} \\ \hline
TexnicCenter & 2.02 &  Można pobrać 32 lub 64 bitową wersję & \url{http://www.texniccenter.org/download/} \\ \hline
SumatraPDF & 3.1.1 & Można pobrać 32 lub 64 bitową wersję & \url{http://www.sumatrapdfreader.org/download-free-pdf-viewer.html} \\ \hline
JabRef & 3.3 & Można pobrać 32 lub 64 bitową wersję & \url{http://www.fosshub.com/JabRef.html} \\ \hline
\end{tabularx}
\end{table}
Nic nie stoi jednak na przeszkodzie, aby szablon ten dostosować do wykorzystania z użyciem innych narzędzi. Dostosowanie to mogłoby polegać na zmianie kodowania plików i korekcie deklaracji kodowania znaków w dokumencie głównym (co opisano dalej).

W kodzie źródłowym szablonu zamieszczono komentarze z uwagami pozwalającymi lepiej zrozumieć znaczenie używanych komend. Komentarze te nie są widoczne w pliku \texttt{Dokument.pdf}, który powstaje jako wynik kompilacji szablonu. 


%%%2. środowisko do pisania kodu latexa: 
%%%( )
%%%3. viewer pdf-ów, pozwalający na nawigację zwrotną: Sumatra PDF 3.0
%%%(http://www.sumatrapdfreader.org/download-free-pdf-viewer.html)
%%%
%%%- o konfiguracji texniccenter do współdziałania z sumatra pdf można poczytać sobie na stronie:
%%%http://tex.stackexchange.com/questions/116981/how-to-configure-texniccenter-2-0-with-sumatra-2013-2014-2015-version
%%%(można znaleźć też inne tutoriale)
%%%
%%%4. środowisko do zarządzania bibliografią: JabRef
%%%(http://jabref.sourceforge.net/download.php)
%%%
%%%Polecam też instalację pod windowsami następujących narzędzi:
%%%- Sumatra PDF - przeglądarka pdf umożliwiająca nawigację pomiędzy
%%%edytowanym tekstem a przeglądanym dokumentem (podglądanie tekstu w
%%%TeXnicCenter umieszcza kursor w odpowiednim miejscu w pdfie, podwójne
%%%kliknięcie w pdfie ustawia kursor w edytorze tekstu).
%%%- JabRef - narzędzie do przygotowywania bibliografii.
%%%
%%%
%%%Uwaga: tytuł powinien zmieścić się w okienku kolorowej okładki (którą
%%%powinna dostarczyć uczelniana administracja). Proszę posterować
%%%parametrami, aby "wpasować" w okienko własny tekst.
%%%
%%%Do ASAPa należy wprowadzić pracę dyplomową/projekt inżynierski w pliku o nazwie:
%%%
%%%W04_[nr albumu]_[rok kalendarzowy]_[rodzaj pracy] (szczegółowa instrukcja pod adresem asap.pwr.edu.pl)
%%%
           %%%Przykładowo:
        %%%­W04_123456_2015_praca inżynierska.pdf     - praca dyplomowa inżynierska
        %%%W04_123456_2015_projekt inżynierski.pdf   - projekt inżynierski
        %%%W04_123456_2015_praca magisterska.pdf  - praca dyplomowa magisterska
%%%
              %%%rok kalendarzowy ? rok realizacji kursu „Praca dyplomowa” (nie rok obrony) +). Druga, jeśli zostanie zastosowana, pozwala określić, które z~plików zostaną skompilowane w całości (na przykład kod źródłowy pierwszego i drugiego rozdziału \verb+\includeonly{rozdzial01.tex,rozdzial02.tex}+).
Brak nazwy pliku na liście w poleceniu \verb+\includeonly+ przy jednoczesnym wystąpieniu jego nazwy w poleceniu \verb+\include+ oznacza, że w kompilacji zostaną uwzględnione referencje wygenerowane dla tego pliku wcześniej, sam zaś kod źródłowy pliku nie będzie kompilowany. 

W szablonie wykorzystano klasę dokumentu \texttt{memoir} oraz wybrane pakiety. Podczas kompilacji szablonu w \texttt{MikTeXu} wszelkie potrzebne pakiety zostaną zainstalowane automatycznie (jeśli \texttt{MikTeX} zainstalowano z opcją dynamicznej instalacji brakujących pakietów). W przypadku innych dystrybucji latexowych może okazać się, że pakiety te trzeba doinstalować ręcznie (np.\ pod linuxem z \texttt{TeXLive} trzeba doinstalować dodatkową zbiorczą paczkę, a jeśli ma się menadżera pakietów latexowych - to pakiety latexowe można instalować indywidualnie).

Jeśli w szablonie będzie wykorzystany indeks rzeczowy, kompilację źródeł trzeba będzie rozszerzyć o kroki potrzebne na wygenerowanie plików pośrednich \texttt{Dokument.idx} oraz \texttt{Dokument.ind} oraz dołączenia ich do finalnego dokumentu (podobnie jak to ma miejsce przy generowaniu wykazu literatury).
Szczegóły dotyczące generowania indeksu rzeczowego opisano w podrozdziale~\ref{sec:indeks}.